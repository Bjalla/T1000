%!TEX root = ../dokumentation.tex

\chapter{Einleitung}
Im Laufe des ersten Studienjahres sollen die Studenten einen Einblick in die Programmierung bekommen sowie grundlegende logische und mathematische Fähigkeiten erlangen. Während der in dieser Praxisarbeit erläuterten Praxisphase hatte ich die Möglichkeit, neue Problemlösungsstrategien und Vorgehensweisen kennenzulernen. Außerdem konnte ich eine neue Programmiersprache sowie eine mir bisher unbekannte Art der Programmierung erlernen.

Thema dieser Arbeit ist die Entwicklung und Umsetzung eines Datencrawlers, welcher Daten von einer Social Media-Plattform sammelt und in eine zur Weiterverarbeitung geeignete Form bringt. Hierzu wurde als Programmiersprache Node.js verwendet, in welcher das Konzept der asynchronen Programmierung Anwendung findet. 

Datencrawler sind Werkzeuge, die häufig bei Suchmaschinen als Grundlage zur Sammlung von Daten eingesetzt werden. Eine ähnliche Funktion erhalten sie auch in dem Kontext dieser Arbeit.

%\begin{wrapfigure}{r}{.4\textwidth}
%\centering
%\includegraphics[height=.35\textwidth]{ibm.png}
%\vspace{-15pt}
%\caption{Das Logo der Musterfirma\footnotemark}
%\end{wrapfigure}
%Quelle muss in Fußnote stehen (da sonst aufgrund eines Fehlers nicht kompiliert
% wird)

