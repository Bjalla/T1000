%!TEX root = ../dokumentation.tex

\pagestyle{empty}

\iflang{de}{%
% Dieser deutsche Teil wird nur angezeigt, wenn die Sprache auf Deutsch eingestellt ist.
\renewcommand{\abstractname}{\langabstract} % Text für Überschrift

% \begin{otherlanguage}{english} % auskommentieren, wenn Abstract auf Deutsch sein soll
\begin{abstract}
Webcrawler sind Programme, die vor allem bei Suchmaschinen zum Einsatz kommen, um Daten zu sammeln und zu sortieren. In dieser Arbeit soll es um die Erstellung eines Crawlers gehen, der Social-Media-Beiträge abruft und verarbeitet. Dazu wird zuerst die dafür verwendete Social-Media-Plattform beleuchtet, sowie Grundlagen zum XML-Format, in dem die Daten bereitgestellt werden. Außerdem wird Node.js mit seinen Konzepten und Arbeitsweisen vorgestellt und schließlich der Crawler und seine Funktionen selbst erklärt.
% \end{otherlanguage} % auskommentieren, wenn Abstract auf Deutsch sein soll
}
\end{abstract}